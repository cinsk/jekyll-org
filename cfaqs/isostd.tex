% $Id$
\chapter{ISO C Standard Consideration}
% chapter 22: ISO C Standard Consideration

\begin{rawhtml}
<pre>$Id$</pre>
\end{rawhtml}

\begin{faq}
\Q{22.1} Terminology

\A 
\begin{description}
        \item[access] 
        \verb+<+execution-time action\verb+>+
        to read or to modify the
        value of an object.
        \begin{description}
                \item[Note 1] Where only noe of these two actions is
                meant, ``read'' or ``modify'' is used. 
                \item[Note 2] ``Modify'' includes the case where the new value
                being stored is the same as the previous value.
                \item[Note 3] Expressions that are not evaluated do not access 
                object.
        \end{description}

        \item[alignment] requirement that objects of a particular type be
        located on storage boundaries with addresses that are particular
        multiples of a byte address

        \item[argument] actual argument or actual parameter(deprecated).
        expression in the comma-separated list bounded by the parentheses
        in a function call expression, or a sequence of preprocessing tokens in
        the comma-separated list bounded by the parentheses in a function-like
        macro invocation.

        \item[behavior] external appeareance or action

        \item[implementation-defined behavior]
        unspecified behavior where each implementation documents how the
        choice is made.
        \begin{description}
                \item[Example] An example of implementation-defined behavior is
                        the propagation of the high-order bit when a signed 
                        integer is shifted right.
        \end{description}
        
        \item[locale-specific behavior] behavior that depends on local
        conventions of nationality, culture, and language that each implementation
        documents.
        \begin{description}
                \item[Example] An example of locale-specific behavior is whether 
                the \TT{islower} function returns true for characters other than 
                the 26 lowercase Latin characters.
        \end{description}

        \item[undefined behavior] 
        Behavior, upon use of a nonportable or erroneous program construct or 
        of erroneous data, for which this International Standard imposes
        no requirements.

        \begin{description}
                \item[Note] Possible undefined behavior ranges from ignoring 
                the situation completely with unpredictable results, to behaving 
                during translation or program execution in a documented manner 
                characteristic of the environment (with or without the issuance 
                of a diagnostic message), to terminating a translation or 
                execution (with the issuance of a diagnostic message).
                \item[Example] An example of undefined behavior is the behavior 
                on integer overflow.
        \end{description}

        \item[unspecified behavior]
        Behavior where this International 
        Standard provides two or more possibilities and imposes no further 
        requirements on which is chosen in any instance.
        \begin{description}
                \item[Example] An example of unspecified behavior is the order 
                in which the arguments to a function are evaluated.
        \end{description}

        \item[bit]
        Unit of data storage in the execution environment large enough to hold
        an object that may have one of two values.
        \begin{description}
                \item[Note] It need not be possible to express the address of each individual
        bit of an object.

        \item[byte]
        addressable unit of data storage large enough to hold any member of
        the basic character set of the execution environment.
        NOTE 1 It is possible to express the address of each individual byte of
        an object uniquely.
        NOTE 2 A byte is composed of a contiguous sequence of bits, the number of
        which is implementation-defined. The least significant bit is called
        the \EM{low-order bit}; the most significant bit is called the 
        \EM{high-order bit}.

        \item[character]
        \verb+<+abstract\verb+>+
        member of a set of elements used for the organization, control,
        or representation of data

        \item[character]
        single-byte character 
        \verb+<+C\verb+>+ $<$C$>$ bit representation that fits in a byte

        \item[multibyte character]
        sequence of one or more bytes representing a member of the extended
        chracter set of either the source or the execution environment.
        \begin{description}
                \item[NOTE] The extended character set is a superset of the
                basic character set.
        \end{description}
      \end{description}
\end{description}
\end{faq}

%
% Local Variables:
% coding: euc-kr
% fill-column: 78
% End:
%
